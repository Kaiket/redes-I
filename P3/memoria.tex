\documentclass[a4paper, 11pt]{article}	%Tipo de documento y opciones.

\usepackage[spanish]{babel}				%Idioma en el que se va a escibir.
\usepackage[utf8]{inputenc}				%Reconocimiento de caracteres como tildes.
\usepackage{fancyhdr}					%Paquete usado para la cabecera y el pie de página.

%Creamos comandos para los autores.
\newcommand{\enriquename}{Enrique Cabrerizo Fernández}
\newcommand{\guillermoname}{Guillermo Ruiz Álvarez}

\title{Práctica 3\\Redes de computadores}						%Título.
\author{\enriquename \and \guillermoname}						%Autores.
\date{14/11/2013}												%Fecha.

\pagestyle{fancyplain}					%Estilo: Usar cabeceras.
\fancyhf{}								%Borrar formato estándar.
\lhead{ \fancyplain{}{\enriquename} }	%Nombre de Enrique a la izquierda de la cabecera.
\rhead{ \fancyplain{}{\guillermoname} }	%Nombre de Guillermo a la derecha de la cabecera.
\cfoot{ \fancyplain{}{\thepage} }		%Número de la página centrada en el pie.

\begin{document}		%Comienza documento.
\maketitle			%Página de título.
\newpage				%Salto de página.
\tableofcontents		%Página de índice.
\newpage				%Salto de página.

\section{Introducción}	%Sección de introducción
En esta prática se va a implementar un programa que analizará y caracterizará una captura de paquetes de red. 

El análisis de los paquetes se realizará utilizando un fichero que contentga una traza o directamente una interfaz especificada, dependiendo del argumento utilizado. Ver


%Apendice
\newpage
\appendix
\renewcommand\appendixname{Anexo}
\section{Manual de utilización del programa}
En esta sección se ofrece una breve explicación sobre la utilización del programa implementado.
\subsection{Compilación}
Para compilar el programa se proporciona un fichero Makefile, existen tres opciones equivalentes para la compilación del mismo utilizando el programa make:
\begin{itemize}
\item \textbf{make all} compila el programa y le da el nombre \textit{practica3}
\item \textbf{make practica3} compila el programa y le da el nombre \textit{practica3}
\item \textbf{make main} compila el programa y le da el nombre \textit{main}
\end{itemize}

\subsection{Ejecución}
Para ejecutar el programa se utiliza la siguiente estructura:

\begin{center}
\textbf{./practica3 INTERF $[<$filtro$>$ $<$dato a filtrar$>]$}
\end{center}

\noindent Donde:\\
\textbf{INTERF} es el fichero pcap o interfaz ethernet (ethX con $X \in [0,9]$).\\
\textbf{$[<$filtro$>$ $<$dato a filtrar$>]:$} puede ser:\\
\indent -ipo x.x.x.x : filtro de IP de origen x.x.x.x ($x \in [0, 255]$)\\
\indent -ipd x.x.x.x : filtro de IP de destino x.x.x.x ($x \in [0, 255]$)\\
\indent -po x : filtro de puerto de origen x ($x \in [0,65536]$)\\
\indent -pd x : filtro de puerto de destino x ($x \in [0,65536]$)\\
\indent -etho xx:xx:xx:xx:xx:xx : filtro de MAC origen ($x \in [00,FF]$)\\
\indent -ethd xx:xx:xx:xx:xx:xx : filtro de MAX destino ($x \in [00,FF]$)\\

Se pueden aplicar varios filtros a la vez y el orden de los mismos no se tiene en cuenta.
Si un filtro IP es 0.0.0.0, un filtro de puertos es 0, o un filtro ethernet es 00:00:00:00:00:00 se considerará inexistente, es decir, no se aplicará dicho filtro.

\end{document}